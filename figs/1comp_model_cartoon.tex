\documentclass{article}

\usepackage[latin1]{inputenc}
\usepackage{tikz}
\usetikzlibrary{shapes,arrows}

%%%<
\usepackage{verbatim}
\usepackage[active,tightpage]{preview}
\PreviewEnvironment{tikzpicture}
\setlength\PreviewBorder{5pt}%
%%%>

\begin{comment}
:Title: Simple flow chart
:Tags: Diagrams

With PGF/TikZ you can draw flow charts with relative ease. This flow chart from [1]_
outlines an algorithm for identifying the parameters of an autonomous underwater vehicle model. 

Note that relative node
placement has been used to avoid placing nodes explicitly. This feature was
introduced in PGF/TikZ >= 1.09.

.. [1] Bossley, K.; Brown, M. & Harris, C. Neurofuzzy identification of an autonomous underwater vehicle `International Journal of Systems Science`, 1999, 30, 901-913 
\end{comment}

\begin{document}
\pagestyle{empty}


% Define block styles
\tikzstyle{block} = [rectangle, draw, fill=blue!20, 
    text width=5em, text centered, rounded corners, minimum height=4em]
\tikzstyle{line} = [draw, -latex']
\tikzstyle{textitem} = [rectangle, draw, fill=blue!20, 
    text width=5em, text centered, rounded corners, minimum height=4em]


\begin{tikzpicture}[node distance = 2.5cm, auto]
    % Place nodes
    \node [block] (c1) {$C_1(t)$};
    \node [left of=c1] (u) {$u(t)$};
    \node [coordinate, right of=c1] (out) {};
    % Draw edges
    \path [line] (u) -- node {$K_1$} (c1);
    \path [line] (c1) -- node {$k_2$} (out);
\end{tikzpicture}

\end{document}
