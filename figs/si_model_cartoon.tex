\documentclass{article}

\usepackage[latin1]{inputenc}
\usepackage{tikz}
\usetikzlibrary{shapes,arrows}

%%%<
\usepackage{verbatim}
\usepackage[active,tightpage]{preview}
\PreviewEnvironment{tikzpicture}
\setlength\PreviewBorder{5pt}%
%%%>

%\begin{comment}
%:Title: Simple flow chart
%:Tags: Diagrams
%
%With PGF/TikZ you can draw flow charts with relative ease. This flow chart from [1]_
%outlines an algorithm for identifying the parameters of an autonomous underwater vehicle model. 
%
%Note that relative node
%placement has been used to avoid placing nodes explicitly. This feature was
%introduced in PGF/TikZ >= 1.09.
%
%.. [1] Bossley, K.; Brown, M. & Harris, C. Neurofuzzy identification of an autonomous underwater %vehicle `International Journal of Systems Science`, 1999, 30, 901-913 
%\end{comment}

\begin{document}
\pagestyle{empty}

% Define block styles
\tikzstyle{block} = [rectangle, draw, fill=blue!20, 
    text width=5em, text centered, rounded corners, minimum height=4em]
\tikzstyle{line} = [draw, -latex']

\begin{tikzpicture}[node distance = 3cm, auto]
    % Place nodes
    \node [block] (c1) {$S(t)$};
    \node [coordinate] (out1) at +(0.0,-2.0) {};
    \node [left of=c1] (u) {$N(t)$};
    \node [block, right of=c1] (c2) {$I(t)$};
    \node [coordinate] (out2) at +(3.0,-2.0) {};
    % Draw edges
    \path [line] (u) -- node {$\mu$} (c1);
    \path [line] (c1) -- node {$\nu$} (out1);
    \path [line] (c1) -- node {$\beta I/N$} (c2);
    \path [line] (c2) -- node {$\nu$} (out2);
\end{tikzpicture}

\end{document}
